\paragraph{Class Player}\mbox{}
\begin{tabularx}{\textwidth}{|c||l|l|l|X|}
    \hline
    \cellcolor{lightgray}Class Name & \multicolumn{4}{X|}{Player}\\
    \hline
    \cellcolor{lightgray}Inherits From & \multicolumn{4}{X|}{None}\\
    \hline
    \cellcolor{lightgray}Description & \multicolumn{4}{p{12cm}|}{The abstract class for all players. (Spymasters and Operatives)}\\
    \hline\hline
    
    \cellcolor{lightgray}Attributes & \cellcolor{lightgray}Visibility & \cellcolor{lightgray}Data type & \cellcolor{lightgray}Name & \cellcolor{lightgray}Description\\\cline{2-5}
    \cellcolor{lightgray} & Private & CardType & team & The player's team (red or blue).\\ 
    \hline\hline
    
    \cellcolor{lightgray}Methods & \cellcolor{lightgray}Visibility & \multicolumn{2}{l|}{\cellcolor{lightgray}Method Name} & \cellcolor{lightgray}Description\\\cline{2-5}
    \hline
    \cellcolor{lightgray} & Public & \multicolumn{2}{l|}{Player(CardType team)} & Constructor. Sets the player's team.\\
    \hline
    \cellcolor{lightgray} & Public & \multicolumn{2}{l|}{getTeam()} & Returns the player's team.\\
    \hline
    \cellcolor{lightgray} & Public & \multicolumn{2}{l|}{makeMove(Clue clue, Bipartite bipartite)} & Implemented by Operatives and Spymasters. Operatives return a card, Spymasters return a clue.\\
    \hline
\end{tabularx}